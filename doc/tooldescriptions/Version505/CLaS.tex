\documentclass[conference]{IEEEtran}

\usepackage{xspace}
\usepackage{hyperref}
\newcommand{\todo}[1]{$\langle\!\langle$\marginpar[\raggedleft$\triangleright\triangleright\triangleright$]{$\triangleleft\triangleleft\triangleleft$}\textsf{#1}$\rangle\!\rangle$}
\def\CC{{C\nolinebreak[4]\hspace{-.05em}\raisebox{.4ex}{\tiny\bf ++}}}

\usepackage{color}

\begin{document}
	
% paper title
\title{\textsc{CLaS} -- A Parallel SAT Solver that Combines CDCL, Look-Ahead and SLS Search Strategies}

% author names and affiliations
% use a multiple column layout for up to three different
% affiliations

\author{
\IEEEauthorblockN{Adrian Balint}
\IEEEauthorblockA{Universit{\"a}t Ulm, Germany}
\and
 \IEEEauthorblockN{Davide Lanti}
 \IEEEauthorblockA{Free University of Bolzano, Italy}
 \and
 \IEEEauthorblockN{Ahmed Irfan}
 \IEEEauthorblockA{ Fondazione Bruno Kessler, Italy
}
\IEEEauthorblockA{University of Trento, Italy}
 \and
\IEEEauthorblockN{Norbert Manthey}
\IEEEauthorblockA{TU Dresden, Germany}
}

% conference papers do not typically use \thanks and this command
% is locked out in conference mode. If really needed, such as for
% the acknowledgment of grants, issue a \IEEEoverridecommandlockouts
% after \documentclass

% use for special paper notices
%\IEEEspecialpapernotice{(Invited Paper)}

\def\cp{\textsc{Coprocessor}\xspace}
\def\glucose{\textsc{Glucose~2.2}\xspace}
\def\minisat{\textsc{Minisat~2.2}\xspace}
\def\riss{\textsc{Riss}\xspace}
\def\pcasso{\textsc{PCasso}\xspace}
\def\sparrow{\textsc{Sparrow}\xspace}
\def\scp{\textsc{Sparrow+CP3}\xspace}
\def\str{\textsc{SparrowToRiss}\xspace}
\def\clas{\textsc{CLaS}\xspace}

\definecolor{midgrey}{rgb}{0.5,0.5,0.5}
\definecolor{darkred}{rgb}{0.7,0.1,0.1}
\newcommand{\nnote}[1]{$[$\textcolor{darkred}{norbert}:~~\emph{\textcolor{midgrey}{#1}}$]$}

\maketitle

% the abstract is optional
\begin{abstract}
\clas is a parallel solver that executes the SLS solver \textsc{Sparrow} in parallel to the parallel search space partitioning SAT solver \textsc{Pcasso}. 
Since \textsc{Pcasso} runs a CDCL solver in parallel to its search space partitioning, this solver represents a portfolio of an SLS solver, a CDCL solver, and a look-ahead based search space partitioning solver. 
Yet, no information is exchanged between \textsc{Sparrow} and \textsc{Pcasso}. 
\end{abstract}

\section{Introduction}

Portfolio SAT solvers are a robust way to solve the diverse formulas that arise from the huge range of applications of SAT, and the crafted instances. 
In \textsc{CLaS}, three solving approaches are combined: CDCL, look ahead and SLS. 
The solver \scp~\cite{SparrowToRiss} showed a good performance on hard combinatorial benchmarks in the SAT Competition 2013, and \pcasso~\cite{Pcasso2014} is a very robust parallel search space partitioning solver. 
Hence, we combine the two solving approaches in parallel. 

\section{Main Techniques}

\scp uses the same configuration as the submitted sequential solver~\cite{SparrowCP2014}. 
Before \sparrow is executed, \cp is used to simplify the formula. 
One core of the CPU is reserved for \scp.

The remaining 11 cores are used for \textsc{Pcasso}, which also uses \cp to simplify the input formula. 
The simplification techniques are the same as for the SAT solver \riss as submitted to the sequential tracks~\cite{riss427}. 


\section{Implementation details}

The two solvers are executed in parallel by a Python script. 
\sparrow is implemented in C.
%
\textsc{Pcasso} and \textsc{Coprocessor} are implemented in \CC. 
%
All solvers have been compiled with the GCC \CC compiler as  64\,bit binaries.

\newpage

\section{Availability}

The source code of \textsc{CLaS} is available at \url{tools.computational-logic.org} for research purposes. 

\section*{Acknowledgment}
The authors would like to thank Armin Biere for many helpful discussions on formula simplification and the BWGrid \cite{bwgrid} project for providing computational resources to tune \cp. 
This project was partially funded by the Deutsche Forschungsgemeinschaft (DFG) under the number SCHO 302/9-1.
Finally, the authors would like to thank the ZIH of TU Dresden for providing the computational resources to develop, test and evaluate \clas and \pcasso.


\bibliographystyle{IEEEtran}
\bibliography{local}

\end{document}


